\documentclass[a4paper]{article}
\usepackage[T1]{fontenc}
\usepackage[utf8]{inputenc}
\usepackage[italian]{babel}


\begin{document}

	\section{Introduzione al copyright}
	
		\paragraph{Raccontare gli eventi che portarono la nascita degli editti a Venezia (proto-copyright)}
		
		\paragraph{Chi è Johames di Spira?} % Domanda incomprensibile
		
		\paragraph{Qual è uno dei grandi cambiamenti dell'editto di Ann?}
	
	
	\section{Nascita del progetto GNU}

		\paragraph{Chi è Richard Stallman?}: \\
		Nato nel 1953, Stallman è uno dei principali esponenti del movimento del software libero.
		Megli anni '70, Stallman entra a far parte dell'IBM New York Scientific Center, un gruppo di ricerca nell'ambito della computer science. Stallman passò qui l'estate dopo il diploma di scuola superiore sviluppando programmi in Fortran per il calcolo numerico. Al tempo non era ancora un informatico, ovviamente, ma se ne voleva interessare.\\
		Nel 1971 entra ad Harvard, ma si interessa di più al MIT di Boston. Infatti Nel giugno del 1971 al primo anno da studente all'università Harvard, Stallman diventò un programmatore al laboratorio IA (Intelligenza Artificiale) del MIT, assunto da Russ Noftsker come programmatore di sistema. Da qui, prese parte alla comunità degli hacker, lavorando insieme a Richard Greenblatt e Bill Gosper. \\
		Nel 1980, al MIT c'era una stampante(Xerox 9700) al servizio degli utenti. Essa però non aveva un buon driver e alcuni messaggi d'errore non erano comunicati. In particolare, quando essa si inceppava, l'unico modo per scoprirlo era andare a verificare di persona, perdendo del tempo inutilmente. Essendo programmatore di sistema, Stallman voleva fixare questo problema, ma per farlo doveva avere a disposizione il codice sorgente della stampante in questione. Stallman fece richiesta, ma gli venne negato l'accesso al codice sorgente, impedendogli di fixare il bug.\\
		Questo fu un fatto che spinse successivamente Stallman a fondare il progetto GNU(GNU is Not Unix).
		
		\paragraph{Dove aveva cominciato a lavorare prima del MIT?}
		vedi domanda sopra
		\paragraph{Com'è che Stallman entra nella comunità del MIT?}
		vedi domanda sopra
		\paragraph{Qual è il lavoro di Stallman?}
		credo ci si riferisca quando è al MIT, vedi domanda sopra(programmatore di sistema)
		\paragraph{Su cosa comincerà a lavorare Stallman nel MIT?}
		vedi domanda sopra
		\paragraph{Che cos'è TECO?}: \\
		Text Editor and COrrector, è in qualche modo l'antenato di Emacs.\\
		Esso è uno dei primissimo programmi per editing di testo usabili tramite un monitor(e non più tramite telescrivente). Ad essere precisi, TECO è simile a un linguaggio di programmazione interpretato, mirato alla manipolazione del testo. Praticamente ogni carattere è un comando (una sequenza di uno o due caratteri rimpiazza le usuali parole chiave di linguaggi più verbosi) e quindi ogni stringa di caratteri è un programma TECO, anche se non necessariamente un programma utile. È sempre stato considerato molto complicato e, proprio per questo motivo, Stallman decise di cercare un qualche altro editor. \\
		Non trovando risultati soddisfacenti, Stallman scrisse una macro per TECO che permetteva di:
		\begin{itemize}
			\item editare il testo real time(prima non era possibile);
			\item random access editing(non ho ben capito cosa sia, credo sia l'opportunità di modificare qualsiasi punto del testo in tempo costante \textit{O}(1));
			\item aggiungere altre macro.
		\end{itemize}
		Cominciarono a nascere diverse macro per TECO, col fine di migliorarlo.
		Ad un certo punto, però, cominciarono a diventare troppe e disordinate. Guy Steele ebbe in seguito l'idea di fare un po' di ordine, cosa che poi Stallman prese a cuore e continuò. Le macro vennero tra loro omogeneizzate e un disordine del genere non sarebbe più dovuto succedere in futuro. \textbf{da qui parte la storia di Emacs...}
	
		\paragraph{Sotto quale licenza era TECO?}:\\
		non ho trovato nulla per ora
	
		\paragraph{Che cos'è Emacs?} %GNU HURD ???
		Emacs è un editor di testo creato da Stallman. Il precedente editor di testo solitamente era utilizzato TECO. Emacs nasce come un insieme di MACRO per TECO per il miglioramento del display e il realtime random access editing. Poi divenne un editor di testo a sè stante. Inizialmente Emacs era distribuito con una clausola che imponeva che ogni modifica dovessere essere inviata allo sviluppatore principale, in modo tale che, se fosse stata una buona idea, potesse essere resa disponibile a tutti. Ciò permise di creare una comunità attorno a Emacs ma, dall'altro lato, ridusse la libertà di sviluppo.

		\paragraph{Che cosa permetteva di fare Emacs?}
		Emacs, all'inizio, permetteva l'accesso random ai file e permetteva di vedere il testo che veniva inserito in real time.

		\paragraph{Sotto quale licenza venne distribuito Emacs?}
		Emacs è stato rilasciato inizialmente con l'obbligo di inviare allo sviluppatore principale ogni modifica in modo tale che, se fosse stata una buona idea, potesse essere resa disponibile a tutti. Ciò permise di creare una comunità attorno a Emacs ma, dall'altro lato, ridusse la libertà di sviluppo.

		\paragraph{Che relazione c'è tra TECO e Emacs?}
		Emacs nasce come un insieme di MACRO per TECO per il miglioramento del display e il realtime random access editing. Poi divenne un editor di testo a sè stante.
		
		\paragraph{Durante la fine del periodo hacker (anni '70), che cosa fa crollare tutto? Chi e perché si organizza uno sciopero?}
		Durante la fine degli anni '70 inizia la fine del periodo hacker. Una delle prime fratture si creò quando al MIT vennero messe le password per l'accesso ai computer dei laboratori. Stallman cerca di convincere le altre persone a limitarne l’utilizzo, e a permettere agli altri utenti di utilizzare i file di tutti. Ciò porta l’intervento del ministero della Difesa, che obbliga l'utilizzo di password. Ciò aliena la comunità vicino a Stallman, anche a causa dell’introduzione dello sciopero del software: Stallman non vuole concedere le ultime versioni di Emacs al laboratorio dell'MIT finchè non venissero tolti tutti i sistemi di sicurezza. \\
		Successivamente, negli anni '70-'80 ha una frammentaione della cultura hacker, causata dal fatto che gli Hacker originari abbandonano il MIT per lavorare o aprire la propria azienda. Si ha un cambiamento dei visitatori al MIT, e con essi cominciano a essere presenti i primi programmi protetti da copyright nel laboratorio di intelligenza artificiale. \\
		La nascita della lisp machine causa la crisi finale. Dall’idea avuta da Greenblat, viene costruita una macchina concepita per funzionare in sintonia con Lisp, ed ebbe un buon successo. Con i copiosi fondi del progetto, al MIT vennero prodotte 32 macchine, che si voleva far comunicare in rete per favorirne la condivisione. Ciò porta Greenblat all’idea di creare un’azienda “hacker friendly” per la produzione di lisp machine. Greenblat si scontrò con Russel Noftsker, che propone, invece, di creare un’azienda “per azioni” e renderla prettamente commerciale. Viene raggiunto un accordo tra i due: a Greenblat viene dato un anno di tempo per creare un’azienda per la vendita di lisp machines. Allo scadere dell'anno, se Greenblat non cci fosse riuscito allora Nofsker avrebbe creato l'azienda. Greenblat ci ci riesce e crea la LMI ma nonostante ciò viene creata un’altra azienda da Nofsker, la Symbolics. Ciò causò causo lo svuotamento del MIT poichè le due aziende attinsero dal MIT per cercare sviluppatori. Inizialmente le due aziende utilizzavano un sistema operativo comune e concessero al MIT di utilizzarlo, ma nel 1982 la Symbolics introdusse delle modifiche al tala sistema operativo e le rese proprietarie. Ciò causo l'ira di Stallman che si mise a replicare le nuove funzionalità per donarle alla LMI. Ciò non fece altro, però, che indebolire la Symbolics. In tutto questo la comunità hacker si indebolisce sempre più, orfana dei fondatori del movimento e incapace di mantenere l'ITS all'interno dei laboratori del MIT. Il crollo definitivo si ha quando nei laboratori l'ITS, diventato ormai obsoleto e poco sicuro, viene sostituito da Tweenex, un software proprietario.

		\paragraph{Chi ha inventato le Lisp machine? Cosa sono? Quali aziende le hanno inventate? Perché queste aziende hanno avuto un impatto sulla comunità hacker?}
		{per una risposta più completa vedi domanda precedente}
		Le Lisp machine sono state inventate da Greenblat, uno dei fondatori del movimento hacker. Le Lisp machine sono delle macchine concepite al fine di lavorare in sintonia con il linguaggio di programmazione Lisp. Tali macchine sono poi state rivendute da due aziende, l'LMI di Greenblat e la Symbolics di Nofsker. Ciò porterà alla crisi definitiva della comunità hacker causata da:
		\begin{itemize}
			\item Gli hacker del MIT vengono chiamati a lavorare nelle due aziende;
			\item La Symbolics crea software con modifiche non rilasciate all'altra azienda;
			\item I padri del movimento hacker, non più al MIT, non riescono a sostenere il loro software interno e ciò porta all'utilizzo di software proprietario.
		\end{itemize}
	

	\section{Linux}
		
		\paragraph{Racconta la nascita di Linux partendo da il sistema UNIX passando per MINIX}
		
		\paragraph{Come si è evoluto e diffuso Linux?}
	
	
	\section{Open source}
	
		\paragraph{Come è nato il movimento Open-Source?}
		
		\paragraph{Chi scrive "La cattedrale e il bazaar"? Di cosa parla? In quali anni?}
		
		\paragraph{Quale grande evento ha fatto si che Raymond diventasse famoso?}
		
		\paragraph{A quali scopi Netscape rilascia il codice sorgente?}
		
		\paragraph{Cosa relaziona il rilascio del codice sorgente di Netscape a Raymond?}
		
		
	\section{Creative Commons}
	
		\paragraph{Come nasce il primo movimento di Creative Commons?}
			Nata ufficialmente nel 2001 grazie al prof. Lawrence Lessig, il movimento CC nasce come associazione non lucrativa per diffondere la politica di copyleft anche nelle opere non informatiche (nascevano infatti nuovi formati digitali per la trasmissione di opere artistiche). 
			Tutto è stato anticipato dalla licenza GNU Free Documentation License (GFDL) che mirava a proteggere la documentazione software, cosa che non avveniva con la sola GPL. Essa garantisce libertà di modifica e regola la distribuzione per grandi quantità imponendo restrizioni sulle redistribuzioni senza modifica e con.
			% bisogna parlare dei fatti che vengono prima
			% la perseveranza del copyright dagli anni 60;
			% l'incontro con la filosofia di Eldred
			% la volontà di andare contro il Sonny Bono Copyright Extension Act, tale cosa avrebbe tolto la lgee ma non risolto il problema, 
			% il sito creato da Eldred fu l'ispirazione per la creazione di un nuovo movimento
		
		\paragraph{Qualcuno ha guidato il movimento Open Content? Chi sono state le prime persone che si sono battute per questi diritti?}
		Eric Eldred che voleva combattere la legge sul copyright secondo lui troppo restrittiva e Lawrence Lessig che voleva proteggere il bene comune tramite la legge.
		
		
	\section{Licenze}
	
		\subsection{GPL v2}
		
			\paragraph{Illustrare tutti gli obblighi che una persona deve soddisfare per poter distribuire solo i binari (nel testo)}
			
		
			\paragraph{Nel caso di sorgenti modificati quali sono gli obblighi? (Illustrare nel testo)}
		
			\paragraph{Cosa fare nel caso dovessi passare la libreria a una terza persona? Ci sono limitazioni?}
			
			\paragraph{Come rendere disponibile la consultazione dei codici sorgenti di binari con licenza GPL v2?} % 2 possibilità
			
			\paragraph{Quali sono le 3 condizioni che si applicano per chi distribuisce solo i binari?}
			
			\paragraph{Cosa contiene la clausola "Libertà o morte"?}
		
		\subsection{GPL v3}
			
			\paragraph{Illustrare le caratteristiche principali}
		
		\subsection{BSD}
		
			\paragraph{Illustrare brevemente 2a, 3a e 4a clausola}
				La BSD versione 1.0 è costituita da 4 clausole, in seguito usciranno versioni senza la 4a clausola e poi senza la 3a e 4a.
				\begin{description}
					\item[2 clausola] indica che qualsiasi ridistribuzione dei file binari deve riportare il testo della licenza con i tre elementi: nota di copyright, lista delle condizioni, avvisi di garanzia e responsabilità;
					\item[3 clausola] il materiale pubblicitario che interessa il software sotto la licenza deve mostrare il riconoscimento:  "Questo prodotto include software sviluppati dalla <organizzazione>." (la seguente clausola è stata poi tolta perché era una sorta di pubblicità gratuita e interferiva con la libertà del software derivato)
					\item[4 clausola] il nome dell'<organizzazione> e dei collaboratori non può essere usato per scopi commerciali.
				\end{description}
			
		\subsection{Apache}
		
			\paragraph{Spiegare la clausola di rinomina}
				La clausola di rinomina fa sì che la licenza conceda l'uso del software ma non del marchio del software e dell'organizzazione.
				Nel caso Apache, la licenza protegge il nome Apache e Apache Software Foundation per scopi pubblicitari senza il permesso e per l'uso del nome stesso in progetti derivati senza il permesso. Al contrario della BSD il marchio è protetto.
			
	
		\subsection{Mozilla public license}
		
			\paragraph{Spiegare la licenza}
		
		
	\section{Date importanti}
	
	\begin{description}
		\item[1700:] anni degli editti a Venezia (proto-copyright)
		\item[1710:] editto di Ann
		\item[1976:] Copyright UA
		\item[1983:] Manifesto GNU di Stallman
		\item[1991:] Prima versione di Linux
	\end{description}
	
	% --- STRUMENTI OPEN SOURCE --- %
	
	\section{SVN}
	
	\paragraph{Creare un repository SVN}
	
	\paragraph{Creare una working copy on checkout}
	
	\paragraph{Mostrare un ciclo modifica -> commit -> update}
	
	\paragraph{Mostrare l'uso dei comandi svn status e revert <file>}
	
	\paragraph{Quali sono le opzioni di svn status?}
	
	\paragraph{Eseguire il backup del repository}
	
	\paragraph{Cancellare un repository}
	
	\paragraph{Ripristinare il repository}
	
	\paragraph{Mostrare come riconoscere se un file è stato modificato}
	
	\paragraph{Eseguire un commit ignorando alcuni file}
	
	\paragraph{Ignorare tutti i file all'interno di una precisa cartella}
	
	\paragraph{Ottenere conflitti tra due utenti}
	
	\paragraph{Risolvere manualmente un conflitto e committare}
	
	\paragraph{Cancellare le modifiche dopo averle committate}
	
	\paragraph{Caricare il repository in Internet}
	
	\paragraph{Accedere al repository da remoto}
	
	\paragraph{Creare dei permessi per l'accesso al repository}
	
	\paragraph{Copiare il contenuto di una cartella tranne del repository in una cartella branches/stable. Inserire un messaggio}

	\paragraph{Illustrare le conseguenze dell'uso del comando svn update -r}
	
	\paragraph{Mostrare come risolvere l'errore "out-of-date"}	
	

	
	\section{Mercurial} % Lo chiede ???
	
		\paragraph{Creare un repository Mercurial}
		
		\paragraph{Operazioni per configurare un repository Mercurial}
		
		\paragraph{Aggiungere un file al repository e poi commentarlo}	
		
		\paragraph{Eseguire un commit}
		
		\paragraph{Fare una modifica ad un file di testo e committarlo}
		
		\paragraph{Fare revert di una modifica committata}
		
		\paragraph{Rimuovere una cartella e committare}
		
		
	
\end{document}