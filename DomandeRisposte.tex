\documentclass[a4paper]{article}
\usepackage[T1]{fontenc}
\usepackage[utf8]{inputenc}
\usepackage[italian]{babel}


\begin{document}

	\section{Introduzione al copyright}
	
		\paragraph{Raccontare gli eventi che portarono la nascita degli editti a Venezia (proto-copyright)}
		
		\paragraph{Chi è Johames di Spira?} % Domanda incomprensibile
		
		\paragraph{Qual è uno dei grandi cambiamenti dell'editto di Ann?}
	
	
	\section{Nascita del progetto GNU}

		\paragraph{Chi è Richard Stallman?}
	
		\paragraph{Dove aveva cominciato a lavorare prima del MIT?}
	
		\paragraph{Com'è che Stallman entra nella comunità del MIT?}
	
		\paragraph{Qual è il lavoro di Stallman?}
	
		\paragraph{Su cosa comincerà a lavorare Stallman nel MIT?}
	
	
		\paragraph{Che cos'è TECO?}
	
		\paragraph{Sotto quale licenza era TECO?}
	
		\paragraph{Che cos'è Emacs?} %GNU HURD ???
	
		\paragraph{Che cosa permetteva di fare Emacs?}
	
		\paragraph{Sotto quale licenza venne distribuito Emacs?}

		\paragraph{Che relazione c'è tra TECO e Emacs?}
		
		
		\paragraph{Durante la fine del periodo hacker (anni '70), che cosa fa crollare tutto? Chi e perché si organizza uno sciopero?}
		
		
		\paragraph{Chi ha inventato le Lisp machine? Cosa sono? Quali aziende le hanno inventate? Perché queste aziende hanno avuto un impatto sulla comunità hacker?}
	
	

	\section{Linux}
		
		\paragraph{Racconta la nascita di Linux partendo da il sistema UNIX passando per MINIX}
		
		\paragraph{Come si è evoluto e diffuso Linux?}
	
	
	\section{Open source}
	
		\paragraph{Come è nato il movimento Open-Source?}
		
		\paragraph{Chi scrive "La cattedrale e il bazaar"? Di cosa parla? In quali anni?}
		
		\paragraph{Quale grande evento ha fatto si che Raymond diventasse famoso?}
		
		\paragraph{A quali scopi Netscape rilascia il codice sorgente?}
		
		\paragraph{Cosa relaziona il rilascio del codice sorgente di Netscape a Raymond?}
		
		
	\section{Creative Commons}
	
		\paragraph{Come nasce il primo movimento di Creative Commons?}
			Nata ufficialmente nel 2001 grazie al prof. Lawrence Lessig, il movimento CC nasce come associazione non lucrativa per diffondere la politica di copyleft anche nelle opere non informatiche (nascevano infatti nuovi formati digitali per la trasmissione di opere artistiche). 
			Tutto è stato anticipato dalla licenza GNU Free Documentation License (GFDL) che mirava a proteggere la documentazione software, cosa che non avveniva con la sola GPL. Essa garantisce libertà di modifica e regola la distribuzione per grandi quantità imponendo restrizioni sulle redistribuzioni senza modifica e con.
			% bisogna parlare dei fatti che vengono prima
			% la perseveranza del copyright dagli anni 60;
			% l'incontro con la filosofia di Eldred
			% la volontà di andare contro il Sonny Bono Copyright Extension Act, tale cosa avrebbe tolto la lgee ma non risolto il problema, 
			% il sito creato da Eldred fu l'ispirazione per la creazione di un nuovo movimento
		
		\paragraph{Qualcuno ha guidato il movimento Open Content? Chi sono state le prime persone che si sono battute per questi diritti?}
		Eric Eldred che voleva combattere la legge sul copyright secondo lui troppo restrittiva e Lawrence Lessig che voleva proteggere il bene comune tramite la legge.
		
		
	\section{Licenze}
	
		\subsection{GPL v2}
		
			\paragraph{Illustrare tutti gli obblighi che una persona deve soddisfare per poter distribuire solo i binari (nel testo)}
			
		
			\paragraph{Nel caso di sorgenti modificati quali sono gli obblighi? (Illustrare nel testo)}
		
			\paragraph{Cosa fare nel caso dovessi passare la libreria a una terza persona? Ci sono limitazioni?}
			
			\paragraph{Come rendere disponibile la consultazione dei codici sorgenti di binari con licenza GPL v2?} % 2 possibilità
			
			\paragraph{Quali sono le 3 condizioni che si applicano per chi distribuisce solo i binari?}
			
			\paragraph{Cosa contiene la clausola "Libertà o morte"?}
		
		\subsection{GPL v3}
			
			\paragraph{Illustrare le caratteristiche principali}
		
		\subsection{BSD}
		
			\paragraph{Illustrare brevemente 2a, 3a e 4a clausola}
				La BSD versione 1.0 è costituita da 4 clausole, in seguito usciranno versioni senza la 4a clausola e poi senza la 3a e 4a.
				\begin{description}
					\item[2 clausola] indica che qualsiasi ridistribuzione dei file binari deve riportare il testo della licenza con i tre elementi: nota di copyright, lista delle condizioni, avvisi di garanzia e responsabilità;
					\item[3 clausola] il materiale pubblicitario che interessa il software sotto la licenza deve mostrare il riconoscimento:  "Questo prodotto include software sviluppati dalla <organizzazione>." (la seguente clausola è stata poi tolta perché era una sorta di pubblicità gratuita e interferiva con la libertà del software derivato)
					\item[4 clausola] il nome dell'<organizzazione> e dei collaboratori non può essere usato per scopi commerciali.
				\end{description}
			
		\subsection{Apache}
		
			\paragraph{Spiegare la clausola di rinomina}
				La clausola di rinomina fa sì che la licenza conceda l'uso del software ma non del marchio del software e dell'organizzazione.
				Nel caso Apache, la licenza protegge il nome Apache e Apache Software Foundation per scopi pubblicitari senza il permesso e per l'uso del nome stesso in progetti derivati senza il permesso. Al contrario della BSD il marchio è protetto.
			
	
		\subsection{Mozilla public license}
		
			\paragraph{Spiegare la licenza}
		
		
	\section{Date importanti}
	
	\begin{description}
		\item[1700:] anni degli editti a Venezia (proto-copyright)
		\item[1710:] editto di Ann
		\item[1976:] Copyright UA
		\item[1983:] Manifesto GNU di Stallman
		\item[1991:] Prima versione di Linux
	\end{description}
	
	% --- STRUMENTI OPEN SOURCE --- %
	
	\section{SVN}
	
	\paragraph{Creare un repository SVN}
	
	\paragraph{Creare una working copy on checkout}
	
	\paragraph{Mostrare un ciclo modifica -> commit -> update}
	
	\paragraph{Mostrare l'uso dei comandi svn status e revert <file>}
	
	\paragraph{Quali sono le opzioni di svn status?}
	
	\paragraph{Eseguire il backup del repository}
	
	\paragraph{Cancellare un repository}
	
	\paragraph{Ripristinare il repository}
	
	\paragraph{Mostrare come riconoscere se un file è stato modificato}
	
	\paragraph{Eseguire un commit ignorando alcuni file}
	
	\paragraph{Ignorare tutti i file all'interno di una precisa cartella}
	
	\paragraph{Ottenere conflitti tra due utenti}
	
	\paragraph{Risolvere manualmente un conflitto e committare}
	
	\paragraph{Cancellare le modifiche dopo averle committate}
	
	\paragraph{Caricare il repository in Internet}
	
	\paragraph{Accedere al repository da remoto}
	
	\paragraph{Creare dei permessi per l'accesso al repository}
	
	\paragraph{Copiare il contenuto di una cartella tranne del repository in una cartella branches/stable. Inserire un messaggio}

	\paragraph{Illustrare le conseguenze dell'uso del comando svn update -r}
	
	\paragraph{Mostrare come risolvere l'errore "out-of-date"}	
	

	
	\section{Mercurial} % Lo chiede ???
	
		\paragraph{Creare un repository Mercurial}
		
		\paragraph{Operazioni per configurare un repository Mercurial}
		
		\paragraph{Aggiungere un file al repository e poi commentarlo}	
		
		\paragraph{Eseguire un commit}
		
		\paragraph{Fare una modifica ad un file di testo e committarlo}
		
		\paragraph{Fare revert di una modifica committata}
		
		\paragraph{Rimuovere una cartella e committare}
		
		
	
\end{document}